\section{Related Work} \label{sec:relatedwork} This section
briefly summarizes prior work in steganography technologies and
hardware security functions, and discusses how they are related
to the information hiding technique in this paper.


\subsection{Steganography}

With the advent of information technology, digital steganography
has become the subject of considerable study.

A large body of 
work has focused on hiding information within digital
files, such as images, videos, audio files, text, and others
\cite{digitalsurvey2010,stegintro2003,infohidingsurvey}. 
These schemes usually hide data in unused meta-data fields, or 
by exploiting noise in the digital content itself; i.e. altering
colors slightly in an image or frequency components in an
audio file. In all cases the hidden data is tied
to the data in the digital file. A recent proposal \cite{Khan2011}
takes a different approach: using the fragmentation pattern 
of digital files in a file system as a covert channel, 
avoiding tampering with the digital content itself. However, 
hidden data is still innately tied to the existence of a digital
file. Also, modifying
hard drive firmware has been investigated as a potential way to
hide information \cite{harddisksteganography}. Data is hidden in
sectors marked as unusable at the firmware level (instead of the
OS or filesystem level), which renders the sectors inaccessible
to most software and complicates recovery, as it is difficult to
tell legitimately bad sectors from ones used for hiding.

Our proposed scheme for Flash memory shares the concept of exploiting
noise to hide data, in the sense that intentionally created biases
are hidden in inherent variations in Flash program time.
However, unlike the above methods, in which hidden 
information depends upon plainly visible digital files, our 
information hiding scheme uses analog properties of Flash. 
As a result, hidden information
is decoupled from the digital content and instead tied to a
physical object. The use of physical properties makes detecting,
copying, or erasing of hidden information difficult because it
requires detailed and time-consuming analog measurements.

Some steganographic
techniques hide information where it is not encoded in plainly
visible digital files. For example, there exist methods to hide
information in the noise of wireless and optical transmissions
by modifying the physical layer protocol
\cite{802comm,80211ofdmsteg,opticalsteg2009}. 
Our work presents a new way to hide
information in Flash memory. Unlike previous techniques, which
often require special tools or modifications to existing
protocols, the proposed information hiding technique can be
applied to Flash memory chip through a standard interface
without any hardware modification.

To make
the steganographic functions available in the embedded domain,
Stanescu et al. proposed to use an FPGA to efficiently process
steganographic algorithms \cite{steganographyembedded}. Our 
technique gives embedded platforms the ability to hide info 
within the device at a level not visible to the file system,
and requires no additional hardware, as Flash memory is common
on embedded platforms.

%Found a lot of interesting non-file-based steganography

%1. hard drive firmware steganography -- malware takes over hard
%drive firmware, maps some sectors %as "bad" when they are fine,
%and hide data there. OS and even forensic tools cannot see the
%data. %Have to remodify firmware to recover data.

%1. hiding info in the cluster/fragmentation patterns of hdds --
%use fragmented files and their % fragmentation pattern to hide
%1/0 in. data lost if defragmented or deleted. can claim data %
%is indistinguishable from normal fragmentation patterns.

%2. wireless physical layer steganography. use protocol noise
%headroom to send message. %if noise is corrected, message is
%lost. if you know to look in the "noise" then good.

%3. optical network steganography. similar to wireless layer,
%you hide info in the dispersion %of the wavelength...

%. immunochemical steganography (biological... not really
%related to us)

%Also have some work on security of steganography. Unsure how or
%if we should %apply them to our work.

%Provable secure steganography \cite{provablesteg2009}

\subsection{Flash Based Security}

We hide a message in the per-bit program times of Flash memory.
Given the popularity of Flash memory in computing systems, there
have been studies on analog characteristics of Flash memory
\cite{grupp2009}. While we have gained insight from the previous work,
it primarily focuses on using analog variations to build
more efficient computing systems rather than enhancing security.

Recently, there have been proposals to use noise and variations
in Flash memory for security by generating true random numbers
and unique chip fingerprints \cite{trust2011, flash-ieeesp2012}.
We use the partial programming technique that was proposed by
the previous study. However, this paper proposes a completely
new application of Flash memory in the context of information
hiding instead of random number generation and fingerprinting.

\subsection{Physical Unclonable Functions}

Physical Unclonable Functions (PUFs) exploit process variation
to provide unique fingerprints for logic circuits
\cite{suhpuf2007}. Special circuits are built that vary their
output depending on the process variation specific to one
instance of the chip. This work is related to PUFs in the sense
that we exploit physical properties and process variations for
security purposes. However, unlike PUFs, our information
hiding scheme uses process variations to hide information
instead of generating device-specific fingerprints and keys.
Also, our information hiding technique can be applied using
standard Flash chips and does not require any custom circuitry.

